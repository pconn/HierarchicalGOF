\documentclass[12pt,letterpaper,final]{article}\usepackage[]{graphicx}\usepackage[]{color}
%% maxwidth is the original width if it is less than linewidth
%% otherwise use linewidth (to make sure the graphics do not exceed the margin)
\makeatletter
\def\maxwidth{ %
  \ifdim\Gin@nat@width>\linewidth
    \linewidth
  \else
    \Gin@nat@width
  \fi
}
\makeatother

\definecolor{fgcolor}{rgb}{0.345, 0.345, 0.345}
\newcommand{\hlnum}[1]{\textcolor[rgb]{0.686,0.059,0.569}{#1}}%
\newcommand{\hlstr}[1]{\textcolor[rgb]{0.192,0.494,0.8}{#1}}%
\newcommand{\hlcom}[1]{\textcolor[rgb]{0.678,0.584,0.686}{\textit{#1}}}%
\newcommand{\hlopt}[1]{\textcolor[rgb]{0,0,0}{#1}}%
\newcommand{\hlstd}[1]{\textcolor[rgb]{0.345,0.345,0.345}{#1}}%
\newcommand{\hlkwa}[1]{\textcolor[rgb]{0.161,0.373,0.58}{\textbf{#1}}}%
\newcommand{\hlkwb}[1]{\textcolor[rgb]{0.69,0.353,0.396}{#1}}%
\newcommand{\hlkwc}[1]{\textcolor[rgb]{0.333,0.667,0.333}{#1}}%
\newcommand{\hlkwd}[1]{\textcolor[rgb]{0.737,0.353,0.396}{\textbf{#1}}}%
\let\hlipl\hlkwb

\usepackage{framed}
\makeatletter
\newenvironment{kframe}{%
 \def\at@end@of@kframe{}%
 \ifinner\ifhmode%
  \def\at@end@of@kframe{\end{minipage}}%
  \begin{minipage}{\columnwidth}%
 \fi\fi%
 \def\FrameCommand##1{\hskip\@totalleftmargin \hskip-\fboxsep
 \colorbox{shadecolor}{##1}\hskip-\fboxsep
     % There is no \\@totalrightmargin, so:
     \hskip-\linewidth \hskip-\@totalleftmargin \hskip\columnwidth}%
 \MakeFramed {\advance\hsize-\width
   \@totalleftmargin\z@ \linewidth\hsize
   \@setminipage}}%
 {\par\unskip\endMakeFramed%
 \at@end@of@kframe}
\makeatother

\definecolor{shadecolor}{rgb}{.97, .97, .97}
\definecolor{messagecolor}{rgb}{0, 0, 0}
\definecolor{warningcolor}{rgb}{1, 0, 1}
\definecolor{errorcolor}{rgb}{1, 0, 0}
\newenvironment{knitrout}{}{} % an empty environment to be redefined in TeX

\usepackage{alltt}
\usepackage[margin=1in]{geometry}
\usepackage{amsfonts}
\usepackage{amsmath}
\usepackage{amstext}
\usepackage{amsthm}
\usepackage{float}
\usepackage[T1]{fontenc}
\usepackage[hidelinks]{hyperref}
\usepackage{mathtools}
\setlength\parindent{0pt}
\IfFileExists{upquote.sty}{\usepackage{upquote}}{}
\begin{document}
\begin{center}
  \bf {\large A GUIDE TO BAYESIAN MODEL CHECKING FOR ECOLOGISTS}

  \vspace{0.7cm} Paul B. Conn$^{1*}$, Devin S. Johnson$^1$, Perry
  J. Williams$^{2,3}$, Sharon R. Melin$^1$, and Mevin
  B. Hooten$^{4,2,3}$
\end{center}
\vspace{0.5cm}


\rm \small

\it $^1$Marine Mammal Laboratory, Alaska Fisheries Science Center,
NOAA National Marine Fisheries Service,
Seattle, Washington 98115 U.S.A.\\

\it $^2$Department of Fish, Wildlife, and Conservation Biology, Colorado State University, Fort Collins, CO 80523 U.S.A.\\

\it $^3$Department of Statistics, Colorado State University, Fort Collins, CO 80523 U.S.A.\\

\it $^4$U.S. Geological Survey, Colorado Cooperative Fish and Wildlife
Research Unit, Colorado State University, Fort Collins, CO 80523
U.S.A.

\vspace{0.7cm}

\textbf{\Large Appendix C: $N$-mixture models} \\


\tableofcontents

\listoftables

\listoffigures


\pagebreak

\section{Simulation Study}

\subsection{Summary}
This section describes the simulation study used for assessing the
closure assumption of N-mixture models in the manuscript \emph{A guide
  to Bayesian model checking for ecologists}.



\subsection{Setting up the \texttt{R} workspace}
Run the following script to install and load required packages and
functions needed for the analysis.

\begin{knitrout}
\definecolor{shadecolor}{rgb}{0.969, 0.969, 0.969}\color{fgcolor}\begin{kframe}
\begin{alltt}
\hlstd{devtools}\hlopt{::}\hlkwd{install_github}\hlstd{(}\hlstr{"pconn/HierarchicalGOF/HierarchicalGOF"}\hlstd{)}

\hlstd{required.packages}\hlkwb{=}\hlkwd{c}\hlstd{(}\hlstr{"coda"}\hlstd{,}
                    \hlstr{"devtools"}\hlstd{,}
                    \hlstr{"HierarchicalGOF"}\hlstd{,}
                    \hlstr{"mcmcse"}\hlstd{,}
                    \hlstr{"parallel"}\hlstd{,}
                    \hlstr{"purrr"}\hlstd{,}
                    \hlstr{"roxygen2"}\hlstd{,}
                    \hlstr{"xtable"} \hlstd{)}

\hlcom{## install.packages(required.packages)}
\hlkwd{lapply}\hlstd{(required.packages,library,}\hlkwc{character.only}\hlstd{=}\hlnum{TRUE}\hlstd{)}
\end{alltt}
\end{kframe}
\end{knitrout}

\subsection{Simulate data for each value of $c$}

\begin{knitrout}
\definecolor{shadecolor}{rgb}{0.969, 0.969, 0.969}\color{fgcolor}\begin{kframe}
\begin{alltt}
\hlcom{###}
\hlcom{### Simulate data}
\hlcom{###}

\hlstd{no.c.vals}\hlkwb{=}\hlnum{8}
\hlstd{Y.list}\hlkwb{=}\hlkwd{list}\hlstd{()}
\hlstd{c.val}\hlkwb{=}\hlkwd{rep}\hlstd{(}\hlkwd{seq}\hlstd{(}\hlnum{0}\hlstd{,}\hlnum{0.35}\hlstd{,}\hlkwc{length.out}\hlstd{=no.c.vals),}\hlkwc{each}\hlstd{=}\hlnum{2}\hlstd{)}
\hlstd{seed}\hlkwb{=}\hlkwd{rep}\hlstd{(}\hlnum{1}\hlopt{:}\hlstd{no.c.vals,}\hlkwc{each}\hlstd{=}\hlnum{2}\hlstd{)}
\hlstd{n}\hlkwb{=}\hlnum{300}
\hlstd{J}\hlkwb{=}\hlnum{5}
\hlstd{alpha}\hlkwb{=}\hlkwd{c}\hlstd{(}\hlnum{1}\hlstd{,}\hlopt{-}\hlnum{1}\hlstd{)}
\hlstd{w1}\hlkwb{=}\hlkwd{rep}\hlstd{(}\hlnum{1}\hlstd{,n)}
\hlstd{w2}\hlkwb{=}\hlkwd{rbinom}\hlstd{(n,}\hlnum{1}\hlstd{,}\hlnum{0.5}\hlstd{)}
\hlstd{W}\hlkwb{=}\hlkwd{cbind}\hlstd{(w1,w2)}
\hlstd{p}\hlkwb{=}\hlkwd{exp}\hlstd{(W}\hlopt\hlstd{alpha)}\hlopt{/}\hlstd{(}\hlnum{1}\hlopt{+}\hlkwd{exp}\hlstd{(W}\hlopt\hlstd{alpha))}
\hlstd{beta}\hlkwb{=}\hlkwd{c}\hlstd{(}\hlnum{4.5}\hlstd{,}\hlnum{1}\hlstd{)}
\hlstd{x1}\hlkwb{=}\hlkwd{rep}\hlstd{(}\hlnum{1}\hlstd{,n)}
\hlstd{x2}\hlkwb{=}\hlkwd{rbinom}\hlstd{(n,}\hlnum{1}\hlstd{,}\hlnum{0.5}\hlstd{)}
\hlstd{X}\hlkwb{=}\hlkwd{cbind}\hlstd{(x1,x2)}
\hlstd{lambda}\hlkwb{=}\hlkwd{exp}\hlstd{(X}\hlopt\hlstd{beta)}
\hlstd{N.true}\hlkwb{=}\hlkwd{matrix}\hlstd{(,n,J)}
\hlstd{N.true[,}\hlnum{1}\hlstd{]}\hlkwb{=}\hlkwd{rpois}\hlstd{(n,lambda)}
\hlkwa{for}\hlstd{(i} \hlkwa{in} \hlnum{1}\hlopt{:}\hlkwd{length}\hlstd{(c.val))\{}
    \hlkwd{set.seed}\hlstd{(seed[i])}
    \hlkwa{for}\hlstd{(j} \hlkwa{in} \hlnum{2}\hlopt{:}\hlstd{J)\{}
        \hlstd{N.true[,j]}\hlkwb{=}\hlkwd{mapply}\hlstd{(}
            \hlstd{rdunif,} \hlnum{1}\hlstd{,} \hlkwd{round}\hlstd{(N.true[,j}\hlopt{-}\hlnum{1}\hlstd{]}\hlopt{*}\hlstd{(}\hlnum{1}\hlopt{+}\hlstd{c.val[i])),}
            \hlkwd{round}\hlstd{(N.true[,j}\hlopt{-}\hlnum{1}\hlstd{]}\hlopt{*}\hlstd{(}\hlnum{1}\hlopt{-}\hlstd{c.val[i])))}
    \hlstd{\}}
    \hlstd{Y.list[[i]]}\hlkwb{=}\hlkwd{matrix}\hlstd{(}\hlkwd{rbinom}\hlstd{(J}\hlopt{*}\hlstd{n,N.true,p),n,J)}
\hlstd{\}}
\hlstd{starting.values.l}\hlkwb{=}\hlkwd{list}\hlstd{()}
\hlkwa{for}\hlstd{(i} \hlkwa{in} \hlnum{1}\hlopt{:}\hlkwd{length}\hlstd{(c.val))\{}
    \hlkwa{if}\hlstd{((i}\hlopt{/}\hlnum{1}\hlstd{)}\hlopt\hlnum{1}\hlopt{==}\hlnum{0}\hlstd{)\{}
        \hlstd{starting.values.l[[i]]}\hlkwb{=}\hlkwd{list}\hlstd{(alpha}\hlopt{+}\hlnum{0.1}\hlstd{,}
                                    \hlstd{beta}\hlopt{+}\hlnum{0.1}\hlstd{,}
                                    \hlkwd{apply}\hlstd{(Y.list[[i]],}\hlnum{1}\hlstd{,max))}
    \hlstd{\}}
    \hlkwa{if}\hlstd{((i}\hlopt{/}\hlnum{2}\hlstd{)}\hlopt\hlnum{1}\hlopt{==}\hlnum{0}\hlstd{)\{}
        \hlstd{starting.values.l[[i]]}\hlkwb{=}\hlkwd{list}\hlstd{(alpha}\hlopt{-}\hlnum{0.1}\hlstd{,}
                                    \hlstd{beta}\hlopt{-}\hlnum{0.1}\hlstd{,}
                                    \hlkwd{apply}\hlstd{(Y.list[[i]],}\hlnum{1}\hlstd{,max))}
    \hlstd{\}}
\hlstd{\}}

\hlstd{name.l}\hlkwb{=}\hlkwd{list}\hlstd{()}
\hlkwa{for}\hlstd{(i} \hlkwa{in} \hlnum{1}\hlopt{:}\hlkwd{length}\hlstd{(c.val))\{}
    \hlstd{name.l[[i]]}\hlkwb{=}\hlkwd{paste}\hlstd{(}\hlstr{"~/SimExample"}\hlstd{,i,}\hlstr{".RData"}\hlstd{,}
                      \hlkwc{sep}\hlstd{=}\hlstr{""}\hlstd{)}
\hlstd{\}}
\hlkwd{save.image}\hlstd{(}\hlkwd{paste}\hlstd{(}\hlstr{"~/SimulatedData.RData"}\hlstd{,}\hlkwc{sep}\hlstd{=}\hlstr{""}\hlstd{))}
\end{alltt}
\end{kframe}
\end{knitrout}

\subsection{Run MCMC algorithms}
In this example, 16 MCMC algorithms are run in parallel to fit 16
different models (two chains for each of 8 values of $c$) which
requires 16 cores. If 16 cores are not available, models must be fit
sequentially using a smaller number of cores (e.g., by first setting
\texttt{models=1:8}, and saving results to \texttt{MCMCOutput1}, and
then setting \texttt{models=9:16}, and saving results to \texttt{MCMCOutput2}).
This may take a while.

\begin{knitrout}
\definecolor{shadecolor}{rgb}{0.969, 0.969, 0.969}\color{fgcolor}\begin{kframe}
\begin{alltt}
\hlcom{###}
\hlcom{### Run algorithm}
\hlcom{###}

\hlstd{n.iter}\hlkwb{=}\hlnum{10000000}
\hlstd{checkpoint}\hlkwb{=}\hlnum{100000}
\hlstd{thin}\hlkwb{=}\hlnum{1}
\hlstd{models}\hlkwb{=}\hlnum{1}\hlopt{:}\hlnum{16}

\hlstd{MCMCOutput}\hlkwb{=}\hlkwd{run.chain.2pl.list}\hlstd{(models,}
                              \hlstd{Y.list,}
                              \hlstd{X,}
                              \hlstd{W,}
                              \hlstd{n.iter,}
                              \hlstd{checkpoint,}
                              \hlstd{thin,}
                              \hlstd{name.l,}
                              \hlstd{starting.values.l}
                              \hlstd{)}

\hlcom{## Save model fitting results}
\hlkwd{save}\hlstd{(MCMCOutput,}\hlkwc{file}\hlstd{=}\hlkwd{paste}\hlstd{(}\hlstr{"~/MCMCOutput.RData"}\hlstd{,}
                           \hlkwc{sep}\hlstd{=}\hlstr{""}\hlstd{))}
\end{alltt}
\end{kframe}
\end{knitrout}

\subsection{Summarize results}

\begin{knitrout}
\definecolor{shadecolor}{rgb}{0.969, 0.969, 0.969}\color{fgcolor}\begin{kframe}
\begin{alltt}
\hlcom{##}
\hlcom{## Load output and calculate results}
\hlcom{##}

\hlkwd{load}\hlstd{(}\hlkwd{paste}\hlstd{(}\hlstr{"~/SimulatedData.RData"}\hlstd{,}
           \hlkwc{sep}\hlstd{=}\hlstr{""}\hlstd{))}

\hlcom{##}
\hlcom{## Create empty containers for output}
\hlcom{##}

\hlstd{GR.Diag}\hlkwb{=}\hlkwd{numeric}\hlstd{(no.c.vals)}
\hlstd{ESS}\hlkwb{=}\hlkwd{numeric}\hlstd{(no.c.vals)}
\hlstd{Status}\hlkwb{=}\hlkwd{numeric}\hlstd{(no.c.vals)}
\hlstd{Bayes.p}\hlkwb{=}\hlkwd{numeric}\hlstd{(no.c.vals)}
\hlstd{Mean.N}\hlkwb{=}\hlkwd{numeric}\hlstd{(no.c.vals)} \hlcom{# Truth = 50989}
\hlstd{LB}\hlkwb{=}\hlkwd{numeric}\hlstd{(no.c.vals)}
\hlstd{UB}\hlkwb{=}\hlkwd{numeric}\hlstd{(no.c.vals)}

\hlcom{##}
\hlcom{## Calculate summaries for each of no.c.vals model fits with different c values}
\hlcom{##}

\hlkwa{for}\hlstd{(i} \hlkwa{in} \hlnum{1}\hlopt{:}\hlstd{no.c.vals)\{}

    \hlstd{w1}\hlkwb{=}\hlkwd{new.env}\hlstd{()}
    \hlkwd{load}\hlstd{(}\hlkwd{paste}\hlstd{(}\hlstr{"~/SimExample"}\hlstd{, i}\hlopt{*}\hlnum{2}\hlopt{-}\hlnum{1}\hlstd{,} \hlstr{".RData"}\hlstd{,}
               \hlkwc{sep}\hlstd{=}\hlstr{""}\hlstd{),}
         \hlkwc{envir}\hlstd{=w1)}
    \hlstd{w2}\hlkwb{=}\hlkwd{new.env}\hlstd{()}
    \hlkwd{load}\hlstd{(}\hlkwd{paste}\hlstd{(}\hlstr{"~/SimExample"}\hlstd{,}
               \hlstd{i}\hlopt{*}\hlnum{2}\hlstd{,}
               \hlstr{".RData"}\hlstd{,}
               \hlkwc{sep}\hlstd{=}\hlstr{""}\hlstd{),}
         \hlkwc{envir}\hlstd{=w2)}
    \hlstd{status}\hlkwb{=}\hlkwd{sum}\hlstd{(}\hlopt{!}\hlkwd{is.na}\hlstd{(w2}\hlopt{$}\hlstd{out[,}\hlnum{1}\hlstd{]))}
    \hlstd{thin}\hlkwb{=}\hlnum{1000} \hlcom{# large thinning value required due to autocorrelation}
    \hlstd{burn}\hlkwb{=}\hlnum{100000}
    \hlstd{ind}\hlkwb{=}\hlkwd{seq}\hlstd{(burn}\hlopt{+}\hlnum{1}\hlstd{,status,thin)}
    \hlkwd{length}\hlstd{(ind)}
    \hlstd{N.tot1}\hlkwb{=}\hlstd{w1}\hlopt{$}\hlstd{out[ind,}\hlnum{5}\hlstd{]}
    \hlstd{N.tot2}\hlkwb{=}\hlstd{w2}\hlopt{$}\hlstd{out[ind,}\hlnum{5}\hlstd{]}
    \hlstd{Mean.N[i]}\hlkwb{=}\hlkwd{mean}\hlstd{(w1}\hlopt{$}\hlstd{out[ind,}\hlnum{5}\hlstd{])}
    \hlstd{LB[i]}\hlkwb{=}\hlkwd{quantile}\hlstd{(w1}\hlopt{$}\hlstd{out[ind,}\hlnum{5}\hlstd{],}\hlnum{0.025}\hlstd{)}
    \hlstd{UB[i]}\hlkwb{=}\hlkwd{quantile}\hlstd{(w1}\hlopt{$}\hlstd{out[ind,}\hlnum{5}\hlstd{],}\hlnum{0.975}\hlstd{)}

    \hlcom{##}
    \hlcom{## Gelman Rubin Diagnostic}
    \hlcom{##}

    \hlstd{chain1}\hlkwb{=}\hlkwd{mcmc}\hlstd{(w1}\hlopt{$}\hlstd{out[}\hlnum{1}\hlopt{:}\hlstd{status,}\hlnum{1}\hlopt{:}\hlnum{4}\hlstd{])}
    \hlstd{chain2}\hlkwb{=}\hlkwd{mcmc}\hlstd{(w2}\hlopt{$}\hlstd{out[}\hlnum{1}\hlopt{:}\hlstd{status,}\hlnum{1}\hlopt{:}\hlnum{4}\hlstd{])}
    \hlstd{out.list}\hlkwb{=}\hlkwd{mcmc.list}\hlstd{(chain1,chain2)}
    \hlstd{GR.Diag[i]}\hlkwb{=}\hlkwd{gelman.diag}\hlstd{(out.list,}\hlkwc{confidence} \hlstd{=} \hlnum{0.95}\hlstd{,}
                           \hlkwc{transform}\hlstd{=}\hlnum{FALSE}\hlstd{,}\hlkwc{autoburnin}\hlstd{=}\hlnum{TRUE}\hlstd{)[}\hlnum{2}\hlstd{]}

    \hlcom{##}
    \hlcom{## Effective sample size}
    \hlcom{##}

    \hlstd{ESS[i]}\hlkwb{=}\hlkwd{min}\hlstd{(}\hlkwd{ess}\hlstd{(chain1))}
    \hlstd{Status[i]}\hlkwb{=}\hlstd{status}

    \hlcom{##}
    \hlcom{## Bayesian p-value}
    \hlcom{##}

    \hlstd{T.mcmc.chi2}\hlkwb{=}\hlstd{w1}\hlopt{$}\hlstd{out[ind,}\hlnum{6}\hlstd{]}
    \hlstd{T.data.chi2}\hlkwb{=}\hlstd{w1}\hlopt{$}\hlstd{out[ind,}\hlnum{7}\hlstd{]}
    \hlstd{Bayes.p[i]}\hlkwb{=}\hlkwd{sum}\hlstd{(T.mcmc.chi2}\hlopt{>=}\hlstd{T.data.chi2,}\hlkwc{na.rm}\hlstd{=}\hlnum{TRUE}\hlstd{)}\hlopt{/}\hlkwd{length}\hlstd{(ind)}

\hlstd{\}}

\hlcom{##}
\hlcom{## Sampled posterior predictive value}
\hlcom{##}

\hlstd{sppv}\hlkwb{=}\hlkwd{rep}\hlstd{(}\hlnum{NA}\hlstd{,no.c.vals)}
\hlkwa{for}\hlstd{(i} \hlkwa{in} \hlnum{1}\hlopt{:}\hlstd{no.c.vals)\{}
    \hlstd{w1}\hlkwb{=}\hlkwd{new.env}\hlstd{()}
    \hlkwd{load}\hlstd{(}\hlkwd{paste}\hlstd{(}\hlstr{"~/SimExample"}\hlstd{,}
               \hlstd{i}\hlopt{*}\hlnum{2}\hlopt{-}\hlnum{1}\hlstd{,}
               \hlstr{".RData"}\hlstd{,}
               \hlkwc{sep}\hlstd{=}\hlstr{""}\hlstd{),}
         \hlkwc{envir}\hlstd{=w1)}
    \hlstd{w2}\hlkwb{=}\hlkwd{new.env}\hlstd{()}
    \hlkwd{load}\hlstd{(}\hlkwd{paste}\hlstd{(}\hlstr{"~/SimExample"}\hlstd{,}
               \hlstd{i}\hlopt{*}\hlnum{2}\hlstd{,}
               \hlstr{".RData"}\hlstd{,}
               \hlkwc{sep}\hlstd{=}\hlstr{""}\hlstd{),}
         \hlkwc{envir}\hlstd{=w2)}
    \hlstd{(status}\hlkwb{=}\hlkwd{sum}\hlstd{(}\hlopt{!}\hlkwd{is.na}\hlstd{(w2}\hlopt{$}\hlstd{out[,}\hlnum{1}\hlstd{])))}
    \hlstd{thin}\hlkwb{=}\hlnum{100}
    \hlstd{burn}\hlkwb{=}\hlnum{100000}
    \hlstd{ind}\hlkwb{=}\hlkwd{seq}\hlstd{(burn}\hlopt{+}\hlnum{1}\hlstd{,status,thin)}
    \hlkwd{length}\hlstd{(ind)}

    \hlstd{param.vec.id}\hlkwb{=}\hlkwd{sample}\hlstd{(ind,}\hlnum{1}\hlstd{)}
    \hlstd{param.vec}\hlkwb{=}\hlstd{w1}\hlopt{$}\hlstd{out[param.vec.id,}\hlnum{1}\hlopt{:}\hlnum{4}\hlstd{]}
    \hlstd{alpha.sppv}\hlkwb{=}\hlstd{param.vec[}\hlnum{1}\hlopt{:}\hlnum{2}\hlstd{]}
    \hlstd{p.sppv}\hlkwb{=}\hlkwd{exp}\hlstd{(W}\hlopt\hlstd{alpha.sppv)}\hlopt{/}\hlstd{(}\hlnum{1}\hlopt{+}\hlkwd{exp}\hlstd{(W}\hlopt\hlstd{alpha.sppv))}
    \hlstd{beta.sppv}\hlkwb{=}\hlstd{param.vec[}\hlnum{3}\hlopt{:}\hlnum{4}\hlstd{]}
    \hlstd{lambda.sppv}\hlkwb{=}\hlkwd{exp}\hlstd{(X}\hlopt\hlstd{beta.sppv)}
    \hlstd{N.sppv}\hlkwb{=}\hlkwd{matrix}\hlstd{(,n,}\hlnum{1}\hlstd{)}
    \hlstd{reps}\hlkwb{=}\hlnum{1000}
    \hlstd{T.mcmc.sppv.chi2}\hlkwb{=}\hlkwd{numeric}\hlstd{(reps)}
    \hlstd{T.data.sppv.chi2}\hlkwb{=}\hlkwd{numeric}\hlstd{(reps)}
    \hlstd{N.sppv[,}\hlnum{1}\hlstd{]}\hlkwb{=}\hlkwd{rpois}\hlstd{(n,lambda.sppv)}
    \hlkwa{for}\hlstd{(k} \hlkwa{in} \hlnum{1}\hlopt{:}\hlstd{reps)\{}
        \hlstd{N.sppv[,}\hlnum{1}\hlstd{]}\hlkwb{=}\hlkwd{rpois}\hlstd{(n,lambda.sppv)}
        \hlstd{y.sppv}\hlkwb{=}\hlkwd{matrix}\hlstd{(}\hlkwd{rbinom}\hlstd{(J}\hlopt{*}\hlstd{n,N.sppv,p.sppv),n,J)}
        \hlstd{T.mcmc.sppv.chi2[k]}\hlkwb{=}\hlkwd{sum}\hlstd{(}\hlkwd{apply}\hlstd{(y.sppv,}\hlnum{1}\hlstd{,var))}
        \hlstd{T.data.sppv.chi2[k]}\hlkwb{=}\hlkwd{sum}\hlstd{(}\hlkwd{apply}\hlstd{(Y.list[[i]],}\hlnum{1}\hlstd{,var))}
    \hlstd{\}}
    \hlstd{sppv[i]}\hlkwb{=}\hlkwd{sum}\hlstd{(T.mcmc.sppv.chi2}\hlopt{>=}\hlstd{T.data.sppv.chi2)}\hlopt{/}\hlstd{reps}
\hlstd{\}}
\end{alltt}
\end{kframe}
\end{knitrout}

\subsection{Model checking results table for simulated data}


\begin{knitrout}
\definecolor{shadecolor}{rgb}{0.969, 0.969, 0.969}\color{fgcolor}\begin{kframe}
\begin{alltt}
\hlcom{##}
\hlcom{## Create a table of results}
\hlcom{##}

\hlstd{GR}\hlkwb{=}\hlkwd{unlist}\hlstd{(GR.Diag)}
\hlstd{p.value}\hlkwb{=}\hlstd{Bayes.p}
\hlstd{c}\hlkwb{=}\hlkwd{unique}\hlstd{(c.val)}
\hlstd{xtable.data}\hlkwb{=}\hlkwd{cbind}\hlstd{(c,p.value,sppv,Mean.N,LB,UB,GR,ESS)}
\hlkwd{print}\hlstd{(}\hlkwd{xtable}\hlstd{(xtable.data),}
      \hlkwc{include.rownames}\hlstd{=}\hlnum{FALSE}\hlstd{)}
\end{alltt}
\end{kframe}
\end{knitrout}

% latex table generated in R 3.3.2 by xtable 1.8-2 package Wed Apr 19
% 15:31:20 2017 latex table generated in R 3.3.2 by xtable 1.8-2
% package Wed Apr 19 15:34:25 2017
\begin{table}[ht]
  \centering
  \begin{tabular}{cccccccc}
    \hline \hline
    $c$ & p-value & sppv & Abundance (truth=50,989) & 95\% CRI & GR & ESS \\
    \hline
    0.00 & 0.48 & 0.27 & 51,200 & (49,295, 53,481) & 1.00 & 3,420 \\
    0.05 & 0.40 & 1.00 & 60,047 & (56,605, 63,868) & 1.00 & 3,260 \\
    0.10 & 0.00 & 1.00 & 81,299 & (75,223, 89,601) & 1.01 & 3,194 \\
    0.15 & 0.00 & 1.00 & 97,066 & (89,149, 104,360) & 1.13 & 3,199 \\
    0.20 & 0.00 & 0.02 & 117,624 & (108,825, 127,007) & 1.03 & 3,184 \\
    0.25 & 0.00 & 0.01 & 119,397 & (110,477, 125,992) & 1.06 & 3,206 \\
    0.30 & 0.00 & 0.00 & 133,797 & (124,194, 141,117) & 1.10 & 3,195 \\
    0.35 & 0.00 & 0.00 & 139,951 & (133,351, 147,086) & 1.00 & 3,213 \\
    \hline
  \end{tabular}
  \caption[Model checking results]{Results of one simulation for
    examining the effect of
    the closure assumption on model fit. The notation $c$
    represents the maximum proportion of the population that could
    move in or out of a site between $j-1$ and $j$, p-value  is the
    posterior predictive
    p-value using a $\chi$-squared goodness-of-fit statistic, sppv
    is the sampled predictive p-value using the sum of variance
    test statistic, Abundance is the mean of the marginal
    posterior distribution for total abundance at the 300 sites,
    the 95\% CRI are the 95\% credible intervals, GR is the
    multi-variate Gelman-Rubin convergence diagnostic, and ESS is
    the effective sample size of 10,000,000 MCMC iterations.}
  \label{tab:results}
\end{table}


\subsection{Plot MCMC output}
The script used for plotting MCMC output is provided for one
simulation ($c=0$), below.

\begin{knitrout}
\definecolor{shadecolor}{rgb}{0.969, 0.969, 0.969}\color{fgcolor}\begin{kframe}
\begin{alltt}
\hlcom{##}
\hlcom{## Figures c=0}
\hlcom{##}

\hlstd{w1}\hlkwb{=}\hlkwd{new.env}\hlstd{()}
\hlkwd{load}\hlstd{(}\hlkwd{paste}\hlstd{(}\hlstr{"~/SimExample"}\hlstd{,}
           \hlnum{1}\hlstd{,}
           \hlstr{".RData"}\hlstd{,}\hlkwc{sep}\hlstd{=}\hlstr{""}\hlstd{),}
     \hlkwc{envir}\hlstd{=w1)}
\hlstd{w2}\hlkwb{=}\hlkwd{new.env}\hlstd{()}
\hlkwd{load}\hlstd{(}\hlkwd{paste}\hlstd{(}\hlstr{"~/SimExample"}\hlstd{,}
           \hlnum{2}\hlstd{,}
           \hlstr{".RData"}\hlstd{,}\hlkwc{sep}\hlstd{=}\hlstr{""}\hlstd{),}
     \hlkwc{envir}\hlstd{=w2)}

\hlkwd{par}\hlstd{(}\hlkwc{mfrow}\hlstd{=}\hlkwd{c}\hlstd{(}\hlnum{5}\hlstd{,}\hlnum{2}\hlstd{),}
    \hlkwc{mar}\hlstd{=}\hlkwd{c}\hlstd{(}\hlnum{3}\hlstd{,}\hlnum{4}\hlstd{,}\hlnum{0}\hlstd{,}\hlnum{1}\hlstd{))}
\hlkwd{plot}\hlstd{(w1}\hlopt{$}\hlstd{out[ind,}\hlnum{1}\hlstd{],}\hlkwc{type}\hlstd{=}\hlstr{'l'}\hlstd{,}
     \hlkwc{ylim}\hlstd{=}\hlkwd{c}\hlstd{(}\hlkwd{min}\hlstd{(w1}\hlopt{$}\hlstd{out[ind,}\hlnum{1}\hlstd{]),}\hlkwd{max}\hlstd{(}\hlnum{1}\hlstd{,}\hlkwd{max}\hlstd{(w1}\hlopt{$}\hlstd{out[ind,}\hlnum{1}\hlstd{]))),}
     \hlkwc{ylab}\hlstd{=}\hlkwd{expression}\hlstd{(alpha[}\hlnum{0}\hlstd{]))}
\hlkwd{lines}\hlstd{(w2}\hlopt{$}\hlstd{out[ind,}\hlnum{1}\hlstd{],}\hlkwc{col}\hlstd{=}\hlnum{3}\hlstd{)}
\hlkwd{abline}\hlstd{(}\hlkwc{h}\hlstd{=}\hlnum{1}\hlstd{,}\hlkwc{col}\hlstd{=}\hlnum{2}\hlstd{)}
\hlkwd{plot}\hlstd{(}\hlkwd{density}\hlstd{(w1}\hlopt{$}\hlstd{out[ind,}\hlnum{1}\hlstd{]),}
     \hlkwc{xlim}\hlstd{=}\hlkwd{c}\hlstd{(}\hlkwd{min}\hlstd{(w1}\hlopt{$}\hlstd{out[ind,}\hlnum{1}\hlstd{]),}\hlkwd{max}\hlstd{(}\hlnum{1}\hlstd{,}\hlkwd{max}\hlstd{(w1}\hlopt{$}\hlstd{out[ind,}\hlnum{1}\hlstd{]))),} \hlkwc{main}\hlstd{=}\hlstr{''}\hlstd{)}
\hlkwd{abline}\hlstd{(}\hlkwc{v}\hlstd{=}\hlnum{1}\hlstd{,}\hlkwc{col}\hlstd{=}\hlnum{2}\hlstd{)}
\hlkwd{plot}\hlstd{(w1}\hlopt{$}\hlstd{out[ind,}\hlnum{2}\hlstd{],}\hlkwc{type}\hlstd{=}\hlstr{'l'}\hlstd{,}
     \hlkwc{ylim}\hlstd{=}\hlkwd{c}\hlstd{(}\hlkwd{min}\hlstd{(}\hlopt{-}\hlnum{1}\hlstd{,}\hlkwd{min}\hlstd{(w1}\hlopt{$}\hlstd{out[ind,}\hlnum{2}\hlstd{])),}\hlkwd{max}\hlstd{(w1}\hlopt{$}\hlstd{out[ind,}\hlnum{2}\hlstd{])),}
     \hlkwc{ylab}\hlstd{=}\hlkwd{expression}\hlstd{(alpha[}\hlnum{1}\hlstd{]))}
\hlkwd{lines}\hlstd{(w2}\hlopt{$}\hlstd{out[ind,}\hlnum{2}\hlstd{],}\hlkwc{col}\hlstd{=}\hlnum{3}\hlstd{)}
\hlkwd{abline}\hlstd{(}\hlkwc{h}\hlstd{=}\hlopt{-}\hlnum{1}\hlstd{,}\hlkwc{col}\hlstd{=}\hlnum{2}\hlstd{)}
\hlkwd{plot}\hlstd{(}\hlkwd{density}\hlstd{(w1}\hlopt{$}\hlstd{out[ind,}\hlnum{2}\hlstd{]),}
     \hlkwc{xlim}\hlstd{=}\hlkwd{c}\hlstd{(}\hlkwd{min}\hlstd{(}\hlopt{-}\hlnum{1}\hlstd{,}\hlkwd{min}\hlstd{(w1}\hlopt{$}\hlstd{out[ind,}\hlnum{2}\hlstd{])),}\hlkwd{max}\hlstd{(w1}\hlopt{$}\hlstd{out[ind,}\hlnum{2}\hlstd{])),} \hlkwc{main}\hlstd{=}\hlstr{''}\hlstd{)}
\hlkwd{abline}\hlstd{(}\hlkwc{v}\hlstd{=}\hlopt{-}\hlnum{1}\hlstd{,}\hlkwc{col}\hlstd{=}\hlnum{2}\hlstd{)}
\hlkwd{plot}\hlstd{(w1}\hlopt{$}\hlstd{out[ind,}\hlnum{5}\hlstd{],}\hlkwc{type}\hlstd{=}\hlstr{'l'}\hlstd{,}
\hlkwc{ylim}\hlstd{=}\hlkwd{c}\hlstd{(}\hlkwd{min}\hlstd{(}\hlkwd{min}\hlstd{(w1}\hlopt{$}\hlstd{out[ind,}\hlnum{5}\hlstd{]),}\hlkwd{sum}\hlstd{(N.true[,}\hlnum{1}\hlstd{])),}\hlkwd{max}\hlstd{(w1}\hlopt{$}\hlstd{out[ind,}\hlnum{5}\hlstd{])),} \hlkwc{ylab}\hlstd{=}\hlstr{'N'}\hlstd{)}
\hlkwd{lines}\hlstd{(w2}\hlopt{$}\hlstd{out[ind,}\hlnum{5}\hlstd{],}\hlkwc{col}\hlstd{=}\hlnum{3}\hlstd{)}
\hlkwd{abline}\hlstd{(}\hlkwc{h}\hlstd{=}\hlkwd{sum}\hlstd{(N.true[,}\hlnum{1}\hlstd{]),}\hlkwc{col}\hlstd{=}\hlnum{2}\hlstd{)}
\hlkwd{plot}\hlstd{(}\hlkwd{density}\hlstd{(w1}\hlopt{$}\hlstd{out[ind,}\hlnum{5}\hlstd{]),}\hlkwc{main}\hlstd{=}\hlstr{''}\hlstd{,}
\hlkwc{xlim}\hlstd{=}\hlkwd{c}\hlstd{(}\hlkwd{min}\hlstd{(}\hlkwd{min}\hlstd{(w1}\hlopt{$}\hlstd{out[ind,}\hlnum{5}\hlstd{]),}\hlkwd{sum}\hlstd{(N.true[,}\hlnum{1}\hlstd{])),}\hlkwd{max}\hlstd{(w1}\hlopt{$}\hlstd{out[ind,}\hlnum{5}\hlstd{])))}
\hlkwd{abline}\hlstd{(}\hlkwc{v}\hlstd{=}\hlkwd{sum}\hlstd{(N.true[,}\hlnum{1}\hlstd{]),}\hlkwc{col}\hlstd{=}\hlnum{2}\hlstd{)}
\hlkwd{plot}\hlstd{(w1}\hlopt{$}\hlstd{out[ind,}\hlnum{3}\hlstd{],}\hlkwc{type}\hlstd{=}\hlstr{'l'}\hlstd{,}
\hlkwc{ylim}\hlstd{=}\hlkwd{c}\hlstd{(}\hlkwd{min}\hlstd{(}\hlnum{4.5}\hlstd{,}\hlkwd{min}\hlstd{(w1}\hlopt{$}\hlstd{out[ind,}\hlnum{3}\hlstd{])),}\hlkwd{max}\hlstd{(w1}\hlopt{$}\hlstd{out[ind,}\hlnum{3}\hlstd{])),}
\hlkwc{ylab}\hlstd{=}\hlkwd{expression}\hlstd{(beta[}\hlnum{0}\hlstd{]))}
\hlkwd{lines}\hlstd{(w2}\hlopt{$}\hlstd{out[ind,}\hlnum{3}\hlstd{],}\hlkwc{col}\hlstd{=}\hlnum{3}\hlstd{)}
\hlkwd{abline}\hlstd{(}\hlkwc{h}\hlstd{=}\hlnum{4.5}\hlstd{,}\hlkwc{col}\hlstd{=}\hlnum{2}\hlstd{)}
\hlkwd{plot}\hlstd{(}\hlkwd{density}\hlstd{(w1}\hlopt{$}\hlstd{out[ind,}\hlnum{3}\hlstd{]),}\hlkwc{main}\hlstd{=}\hlstr{''}\hlstd{,}
\hlkwc{xlim}\hlstd{=}\hlkwd{c}\hlstd{(}\hlkwd{min}\hlstd{(}\hlnum{4.5}\hlstd{,}\hlkwd{min}\hlstd{(w1}\hlopt{$}\hlstd{out[ind,}\hlnum{3}\hlstd{])),}\hlkwd{max}\hlstd{(w1}\hlopt{$}\hlstd{out[ind,}\hlnum{3}\hlstd{])))}
\hlkwd{abline}\hlstd{(}\hlkwc{v}\hlstd{=}\hlnum{4.5}\hlstd{,}\hlkwc{col}\hlstd{=}\hlnum{2}\hlstd{)}
\hlkwd{plot}\hlstd{(w1}\hlopt{$}\hlstd{out[ind,}\hlnum{4}\hlstd{],}\hlkwc{type}\hlstd{=}\hlstr{'l'}\hlstd{,}
     \hlkwc{ylab}\hlstd{=}\hlkwd{expression}\hlstd{(beta[}\hlnum{1}\hlstd{]))}
\hlkwd{lines}\hlstd{(w2}\hlopt{$}\hlstd{out[ind,}\hlnum{4}\hlstd{],}\hlkwc{col}\hlstd{=}\hlnum{3}\hlstd{)}
\hlkwd{abline}\hlstd{(}\hlkwc{h}\hlstd{=}\hlnum{1}\hlstd{,}\hlkwc{col}\hlstd{=}\hlnum{2}\hlstd{)}
\hlkwd{plot}\hlstd{(}\hlkwd{density}\hlstd{(w1}\hlopt{$}\hlstd{out[ind,}\hlnum{4}\hlstd{]),} \hlkwc{main}\hlstd{=}\hlstr{''}\hlstd{)}
\hlkwd{abline}\hlstd{(}\hlkwc{v}\hlstd{=}\hlnum{1}\hlstd{,}\hlkwc{col}\hlstd{=}\hlnum{2}\hlstd{)}
\end{alltt}
\end{kframe}
\end{knitrout}

\graphicspath{~/Dropbox/Post-Doc/RFiles/ModelChecking/EcologicalMonographs}
\begin{center}
  \begin{figure}[H]
    \includegraphics[width=.9\linewidth,keepaspectratio]{ConvergencePlots1.pdf}
    \caption[Simulated data: $c=0$]{Trace plots and marginal posterior
      distributions of parameters in $N$-mixture model when $c=0$
      (i.e., the population was closed). Using a Gelman-Rubin
      diagnostic, there is no evidence that the model failed to
      converge (GR=1.00), and posterior distributions recovered true
      parameter values well. Both the posterior predictive p-value
      (0.48) and the sampled predictive p-value (0.27) suggested no
      lack of model fit. 10,000,000 MCMC iterations were conducted and
      thinned to every 1,000 iteration.}
  \end{figure}
\end{center}

\begin{center}
  \begin{figure}[H]
    \includegraphics[width=.9\linewidth,keepaspectratio]{ConvergencePlots3.pdf}
    \caption[Simulated data: $c=0.05$]{Trace plots and marginal
      posterior distributions of parameters in $N$-mixture model when
      $c=0.05$ (i.e., up to 5\% of $N_{i,j-1}$ was allowed to leave or
      enter the population at time $j$). Using a Gelman-Rubin
      diagnostic, there is no evidence that the model failed to
      converge (GR=1.00), but posterior distributions did not recover
      true parameter values well. The posterior predictive p-value
      (0.40) suggested no lack of model fit, but the sampled
      predictive p-value (1.00) suggested lack of model
      fit. 10,000,000 MCMC iterations were conducted and thinned to
      every 1,000 iteration.}
  \end{figure}
\end{center}

\begin{center}
  \begin{figure}[H]
    \includegraphics[width=.9\linewidth,keepaspectratio]{ConvergencePlots5.pdf}
    \caption[Simulated data: $c=0.10$]{Trace plots and marginal
      posterior distributions of parameters in $N$-mixture model when
      $c=0.10$ (i.e., up to 10\% of $N_{i,j-1}$ was allowed to leave
      or enter the population at time $j$). Using a Gelman-Rubin
      diagnostic, there is no evidence that the model failed to
      converge (GR=1.01), but posterior distributions did not recover
      true parameter values well. Both the posterior predictive
      p-value (0.00) and the sampled predictive p-value (1.00)
      suggested lack of model fit. 10,000,000 MCMC iterations were
      conducted and thinned to every 1,000 iteration, with a 100,000
      burn-in period.}
  \end{figure}
\end{center}

\begin{center}
  \begin{figure}[H]
    \includegraphics[width=.9\linewidth,keepaspectratio]{ConvergencePlots7.pdf}
    \caption[Simulated data: $c=0.15$]{Trace plots and marginal
      posterior distributions of parameters in $N$-mixture model when
      $c=0.15$ (i.e., up to 15\% of $N_{i,j-1}$ was allowed to leave
      or enter the population at time $j$). The model did not appear
      to converge (Gelman-Rubin diagnostic=1.13). Both the posterior
      predictive p-value (0.00) and the sampled predictive p-value
      (1.00) suggested lack of model fit. 10,000,000 MCMC iterations
      were conducted and thinned to every 1,000 iteration, with a
      100,000 burn-in period.}
  \end{figure}
\end{center}

\begin{center}
  \begin{figure}[H]
    \includegraphics[width=.9\linewidth,keepaspectratio]{ConvergencePlots9.pdf}
    \caption[Simulated data: $c=0.20$]{Trace plots and marginal
      posterior distributions of parameters in $N$-mixture model when
      $c=0.20$ (i.e., up to 20\% of $N_{i,j-1}$ was allowed to leave
      or enter the population at time $j$). Using a Gelman-Rubin
      diagnostic, there is no evidence that the model failed to
      converge (GR=1.03), but posterior distributions did not recover
      true parameter values well. Both the posterior predictive
      p-value (0.00) and the sampled predictive p-value (0.02)
      suggested lack of model fit. 10,000,000 MCMC iterations were
      conducted and thinned to every 1,000 iteration, with a 100,000
      burn-in period.}
  \end{figure}
\end{center}

\begin{center}
  \begin{figure}[H]
    \includegraphics[width=.9\linewidth,keepaspectratio]{ConvergencePlots11.pdf}
    \caption[Simulated data: $c=0.25$]{Trace plots and marginal
      posterior distributions of parameters in $N$-mixture model when
      $c=0.25$ (i.e., up to 25\% of $N_{i,j-1}$ was allowed to leave
      or enter the population at time $j$). Using a Gelman-Rubin
      diagnostic, there is some evidence that the model failed to
      converge (GR=1.06). Both the posterior predictive p-value (0.00)
      and the sampled predictive p-value (0.01) suggested lack of
      model fit. 10,000,000 MCMC iterations were conducted and thinned
      to every 1,000 iteration, with a 100,000 burn-in period.}
  \end{figure}
\end{center}

\begin{center}
  \begin{figure}[H]
    \includegraphics[width=.9\linewidth,keepaspectratio]{ConvergencePlots13.pdf}
    \caption[Simulated data: $c=0.30$]{Trace plots and marginal
      posterior distributions of parameters in $N$-mixture model when
      $c=0.30$ (i.e., up to 30\% of $N_{i,j-1}$ was allowed to leave
      or enter the population at time $j$). The model did not appear
      to converge (Gelman-Rubin diagnostic=1.10). Both the posterior
      predictive p-value (0.00) and the sampled predictive p-value
      (0.00) suggested lack of model fit. 10,000,000 MCMC iterations
      were conducted and thinned to every 1,000 iteration, with a
      100,000 burn-in period.}
  \end{figure}
\end{center}

\begin{center}
  \begin{figure}[H]
    \includegraphics[width=.9\linewidth,keepaspectratio]{ConvergencePlots15.pdf}
    \caption[Simulated data: $c=0.35$]{Trace plots and marginal
      posterior distributions of parameters in $N$-mixture model when
      $c=0.35$ (i.e., up to 35\% of $N_{i,j-1}$ was allowed to leave
      or enter the population at time $j$). Using a Gelman-Rubin
      diagnostic, there is no evidence that the model failed to
      converge (GR=1.00), but posterior distributions did not recover
      true parameter values well. Both the posterior predictive
      p-value (0.00) and the sampled predictive p-value (0.00)
      suggested lack of model fit. 10,000,000 MCMC iterations were
      conducted and thinned to every 1,000 iteration, with a 100,000
      burn-in period.}
  \end{figure}
\end{center}


%%%%%%%%%%%%%%%%%%%%%%%%%%%%%%%%%%%%%%%%%%%%%%%%%%%%%%%%%%%%%%%%%%%%%%%%%%%%%%
%%%%%%%%%%%%%%%%%%%%%%%%%%%%%%%%%%%%%%%%%%%%%%%%%%%%%%%%%%%%%%%%%%%%%%%%%%%%%%
%%%%%%%%%%%%%%%%%%%%%%%%%%%%%%%%%%%%%%%%%%%%%%%%%%%%%%%%%%%%%%%%%%%%%%%%%%%%%%

\newpage
\section{Sea Otters}

\subsection{Summary}
This section describes the sea otter study used for assessing the
closure assumption of N-mixture models in the manuscript \emph{A guide
  to Bayesian model checking for ecologists}.

\subsection{All data}
In what follows, we conduct model checking using 21 sites sea otters
were observed at in Glacier Bay National Park, including one site
where observers noted a violation of the closure assumption. In the
next section, we remove the site and conduct model checking.
\begin{knitrout}
\definecolor{shadecolor}{rgb}{0.969, 0.969, 0.969}\color{fgcolor}\begin{kframe}
\begin{alltt}
\hlkwd{rm}\hlstd{(}\hlkwc{list}\hlstd{=}\hlkwd{ls}\hlstd{())}

\hlstd{name.1}\hlkwb{=}\hlstr{"~/MCMCOutputAllDataChain1.RData"}
\hlstd{name.2}\hlkwb{=}\hlstr{"~/MCMCOutputAllDataChain2.RData"}

\hlcom{## Load Data}
\hlkwd{data}\hlstd{(SeaOtterData)}

\hlcom{## Use all the data including the 19th row of data where sea otters}
\hlcom{## were observed violating the closure assumption}
\hlstd{Y}\hlkwb{=}\hlstd{SeaOtterData}

\hlcom{## Priors}
\hlstd{q.p}\hlkwb{=}\hlnum{1}
\hlstd{r.p}\hlkwb{=}\hlnum{1}
\hlstd{alpha}\hlkwb{=}\hlnum{0.001}
\hlstd{beta}\hlkwb{=}\hlnum{0.001}
\hlstd{n.iter}\hlkwb{=}\hlnum{10000000}
\hlstd{thin}\hlkwb{=}\hlnum{1}
\hlstd{checkpoint}\hlkwb{=}\hlnum{1000000}
\end{alltt}
\end{kframe}
\end{knitrout}

\subsection{Run MCMC algorithm}
Fit a simple $N$-mixture model with no covariates.

\begin{knitrout}
\definecolor{shadecolor}{rgb}{0.969, 0.969, 0.969}\color{fgcolor}\begin{kframe}
\begin{alltt}
\hlcom{## Run algorithm}
\hlkwd{Nmixmcmc}\hlstd{(}\hlkwc{Y}\hlstd{=Y,q.p,r.p,alpha,beta,n.iter,checkpoint,name.1,thin)}
\hlkwd{Nmixmcmc}\hlstd{(}\hlkwc{Y}\hlstd{=Y,q.p,r.p,alpha,beta,n.iter,checkpoint,name.2,thin)}
\end{alltt}
\end{kframe}
\end{knitrout}

\subsection{Summarize results using all data}

\begin{knitrout}
\definecolor{shadecolor}{rgb}{0.969, 0.969, 0.969}\color{fgcolor}\begin{kframe}
\begin{alltt}
\hlcom{## ## Load output and calculate results ##}

\hlstd{w1}\hlkwb{=}\hlkwd{new.env}\hlstd{()}
\hlkwd{load}\hlstd{(}\hlkwd{paste}\hlstd{(}\hlstr{"~/MCMCOutputAllDataChain1.RData"}\hlstd{,} \hlkwc{sep}\hlstd{=}\hlstr{""}\hlstd{),}
     \hlkwc{envir}\hlstd{=w1)}
\hlstd{w2}\hlkwb{=}\hlkwd{new.env}\hlstd{()}
\hlkwd{load}\hlstd{(}\hlkwd{paste}\hlstd{(}\hlstr{"~/MCMCOutputAllDataChain2.RData"}\hlstd{,}
           \hlkwc{sep}\hlstd{=}\hlstr{""}\hlstd{),} \hlkwc{envir}\hlstd{=w2)}
\hlstd{(status}\hlkwb{=}\hlkwd{sum}\hlstd{(}\hlopt{!}\hlkwd{is.na}\hlstd{(w2}\hlopt{$}\hlstd{out[[}\hlnum{1}\hlstd{]][,}\hlnum{1}\hlstd{])))}
\end{alltt}
\begin{verbatim}
## [1] 10000000
\end{verbatim}
\begin{alltt}
\hlstd{thin}\hlkwb{=}\hlnum{1000} \hlcom{# large thinning value required due to autocorrelation}
\hlstd{burn}\hlkwb{=}\hlnum{100000}
\hlstd{ind}\hlkwb{=}\hlkwd{seq}\hlstd{(burn}\hlopt{+}\hlnum{1}\hlstd{,status,thin)}
\hlkwd{length}\hlstd{(ind)}
\end{alltt}
\begin{verbatim}
## [1] 9900
\end{verbatim}
\begin{alltt}
\hlstd{N.tot1}\hlkwb{=}\hlstd{w1}\hlopt{$}\hlstd{out[[}\hlnum{4}\hlstd{]][ind]}
\hlstd{N.tot2}\hlkwb{=}\hlstd{w2}\hlopt{$}\hlstd{out[[}\hlnum{4}\hlstd{]][ind]}
\hlstd{Mean.N}\hlkwb{=}\hlkwd{mean}\hlstd{(N.tot1)}
\hlstd{LB}\hlkwb{=}\hlkwd{quantile}\hlstd{(N.tot1,}\hlnum{0.025}\hlstd{)}
\hlstd{UB}\hlkwb{=}\hlkwd{quantile}\hlstd{(N.tot1,}\hlnum{0.975}\hlstd{)}

\hlcom{##}
\hlcom{## Gelman Rubin Diagnostic}
\hlcom{##}

\hlstd{mcmc1.tmp}\hlkwb{=}\hlkwd{cbind}\hlstd{(w1}\hlopt{$}\hlstd{out[[}\hlnum{1}\hlstd{]][}\hlnum{1}\hlopt{:}\hlstd{status,],w1}\hlopt{$}\hlstd{out[[}\hlnum{3}\hlstd{]][}\hlnum{1}\hlopt{:}\hlstd{status,])}
\hlstd{mcmc2.tmp}\hlkwb{=}\hlkwd{cbind}\hlstd{(w2}\hlopt{$}\hlstd{out[[}\hlnum{1}\hlstd{]][}\hlnum{1}\hlopt{:}\hlstd{status,],w2}\hlopt{$}\hlstd{out[[}\hlnum{3}\hlstd{]][}\hlnum{1}\hlopt{:}\hlstd{status,])}
\hlstd{chain1}\hlkwb{=}\hlkwd{mcmc}\hlstd{(mcmc1.tmp)}
\hlstd{chain2}\hlkwb{=}\hlkwd{mcmc}\hlstd{(mcmc2.tmp)}
\hlstd{out.list}\hlkwb{=}\hlkwd{mcmc.list}\hlstd{(chain1,chain2)}
\hlstd{(GR.Diag}\hlkwb{=}\hlkwd{gelman.diag}\hlstd{(out.list,}\hlkwc{confidence} \hlstd{=} \hlnum{0.95}\hlstd{,}
\hlkwc{transform}\hlstd{=}\hlnum{FALSE}\hlstd{,}\hlkwc{autoburnin}\hlstd{=}\hlnum{TRUE}\hlstd{)[}\hlnum{2}\hlstd{])}
\end{alltt}
\begin{verbatim}
## $mpsrf
## [1] 1.061397
\end{verbatim}
\begin{alltt}
\hlcom{##}
\hlcom{## Effective sample size}
\hlcom{##}

\hlstd{(ESS}\hlkwb{=}\hlkwd{min}\hlstd{(}\hlkwd{ess}\hlstd{(chain1)))}
\end{alltt}
\begin{verbatim}
## [1] 3213.653
\end{verbatim}
\begin{alltt}
\hlcom{##}
\hlcom{## Bayesian p-value}
\hlcom{##}

\hlstd{T.mcmc.chi2}\hlkwb{=}\hlstd{w2}\hlopt{$}\hlstd{out[[}\hlnum{5}\hlstd{]][ind]}
\hlstd{T.data.chi2}\hlkwb{=}\hlstd{w2}\hlopt{$}\hlstd{out[[}\hlnum{6}\hlstd{]][ind]}
\hlstd{(Bayes.p}\hlkwb{=}\hlkwd{sum}\hlstd{(T.mcmc.chi2}\hlopt{>=}\hlstd{T.data.chi2,}\hlkwc{na.rm}\hlstd{=}\hlnum{TRUE}\hlstd{)}\hlopt{/}\hlkwd{length}\hlstd{(ind))}
\end{alltt}
\begin{verbatim}
## [1] 0.04868687
\end{verbatim}
\begin{alltt}
\hlcom{##}
\hlcom{## Sampled posterior predictive value}
\hlcom{##}

\hlstd{n}\hlkwb{=}\hlkwd{dim}\hlstd{(Y)[}\hlnum{1}\hlstd{]}
\hlstd{J}\hlkwb{=}\hlkwd{dim}\hlstd{(Y)[}\hlnum{2}\hlstd{]}
\hlstd{param.vec.id}\hlkwb{=}\hlkwd{sample}\hlstd{(ind,}\hlnum{1}\hlstd{)}
\hlstd{p.sppv}\hlkwb{=}\hlstd{w1}\hlopt{$}\hlstd{out[[}\hlnum{1}\hlstd{]][param.vec.id]}
\hlstd{lambda.sppv}\hlkwb{=}\hlstd{w1}\hlopt{$}\hlstd{out[[}\hlnum{3}\hlstd{]][param.vec.id,]}
\hlstd{N.sppv}\hlkwb{=}\hlstd{w1}\hlopt{$}\hlstd{out[[}\hlnum{2}\hlstd{]][param.vec.id,]}
\hlstd{Expected.Y}\hlkwb{=}\hlkwd{matrix}\hlstd{(N.sppv}\hlopt{*}\hlstd{p.sppv,}\hlkwd{dim}\hlstd{(Y)[}\hlnum{1}\hlstd{],}\hlkwd{dim}\hlstd{(Y)[}\hlnum{2}\hlstd{])}
\hlstd{reps}\hlkwb{=}\hlnum{100000}
\hlstd{T.mcmc.sppv.chi2}\hlkwb{=}\hlkwd{numeric}\hlstd{(reps)}
\hlstd{T.data.sppv.chi2}\hlkwb{=}\hlkwd{numeric}\hlstd{(reps)}
\hlkwa{for}\hlstd{(k} \hlkwa{in} \hlnum{1}\hlopt{:}\hlstd{reps)\{ y.sppv}\hlkwb{=}\hlkwd{matrix}\hlstd{(}\hlkwd{rbinom}\hlstd{(n}\hlopt{*}\hlstd{J,N.sppv,p.sppv),n,J)}
    \hlstd{y.sppv[}\hlkwd{is.na}\hlstd{(Y)]}\hlkwb{=}\hlnum{NA}
    \hlcom{## T.mcmc.sppv.chi2[k]=sum(apply(y.sppv,1,var,na.rm=TRUE))}
    \hlcom{## T.data.sppv.chi2[k]=sum(apply(Y,1,var,na.rm=TRUE))}
    \hlstd{T.mcmc.sppv.chi2[k]}\hlkwb{=}\hlkwd{sum}\hlstd{(((y.sppv}\hlopt{-}\hlstd{Expected.Y)}\hlopt{^}\hlnum{2}\hlstd{)}\hlopt{/}\hlstd{Expected.Y,}\hlkwc{na.rm}\hlstd{=}\hlnum{TRUE}\hlstd{)}
    \hlstd{T.data.sppv.chi2[k]}\hlkwb{=}\hlkwd{sum}\hlstd{(((Y}\hlopt{-}\hlstd{Expected.Y)}\hlopt{^}\hlnum{2}\hlstd{)}\hlopt{/}\hlstd{Expected.Y,}\hlkwc{na.rm}\hlstd{=}\hlnum{TRUE}\hlstd{)}
\hlstd{\}}
\hlstd{(sppv}\hlkwb{=}\hlkwd{sum}\hlstd{(T.mcmc.sppv.chi2}\hlopt{>=}\hlstd{T.data.sppv.chi2)}\hlopt{/}\hlstd{reps)}
\end{alltt}
\begin{verbatim}
## [1] 0.09605
\end{verbatim}
\end{kframe}
\end{knitrout}

\subsection{Plot MCMC output}



\begin{center}
  \begin{figure}[H]
    \includegraphics[width=.9\linewidth,keepaspectratio]{convergencebars.pdf}
    \caption[Convergence and posterior distributions -- all
    data]{Convergence diagnostics and posterior distributions of model
    parameters using all sea otter data.}
  \end{figure}
\end{center}






\begin{center}
  \begin{figure}[H]
    \includegraphics[width=.9\linewidth,keepaspectratio]{expectedobserved1.pdf}
    \caption[Observed vs. expected -- all data]{Observed vs. expected
      counts of sea otters at 21 sites in Glacier Bay, AK, based on
      model fit of complete data set, including the site that violated
      the closure assumption. Black line: posterior distribution of
      expected counts. Red lines: sum of the observed counts across 21
      sites for two observation periods (i.e., sum of column 1 and column 2 of \texttt{SeaOtterData}).}
  \end{figure}
\end{center}


\subsection{Removing the site where the closure assumption was
  violated}

\begin{knitrout}
\definecolor{shadecolor}{rgb}{0.969, 0.969, 0.969}\color{fgcolor}\begin{kframe}
\begin{alltt}
\hlkwd{rm}\hlstd{(}\hlkwc{list}\hlstd{=}\hlkwd{ls}\hlstd{())}

\hlstd{mod.name.1}\hlkwb{=}\hlstr{"~/MCMCOutputModifiedlDataChain1.RData"}
\hlstd{mod.name.2}\hlkwb{=}\hlstr{"~/MCMCOutputModifiedlDataChain2.RData"}

\hlkwd{data}\hlstd{(SeaOtterData)}
\hlstd{Y}\hlkwb{=}\hlstd{SeaOtterData}

\hlcom{## Remove the 19th row of data where sea otters}
\hlcom{## were observed violating the closure assumption}
\hlstd{Y}\hlkwb{=}\hlstd{Y[}\hlopt{-}\hlnum{19}\hlstd{,]}

\hlcom{## Priors}
\hlstd{q.p}\hlkwb{=}\hlnum{1}
\hlstd{r.p}\hlkwb{=}\hlnum{1}
\hlstd{alpha}\hlkwb{=}\hlnum{0.001}
\hlstd{beta}\hlkwb{=}\hlnum{0.001}
\hlstd{n.iter}\hlkwb{=}\hlnum{10000000}
\hlstd{thin}\hlkwb{=}\hlnum{1}
\hlstd{checkpoint}\hlkwb{=}\hlnum{1000000}
\end{alltt}
\end{kframe}
\end{knitrout}

\subsection{Run MCMC algorithm}
Fit a simple $N$-mixture model with no covariates.

\begin{knitrout}
\definecolor{shadecolor}{rgb}{0.969, 0.969, 0.969}\color{fgcolor}\begin{kframe}
\begin{alltt}
\hlcom{## Run algorithm}
\hlkwd{Nmixmcmc}\hlstd{(}\hlkwc{Y}\hlstd{=Y,q.p,r.p,alpha,beta,n.iter,checkpoint,mod.name.1,thin)}
\hlkwd{Nmixmcmc}\hlstd{(}\hlkwc{Y}\hlstd{=Y,q.p,r.p,alpha,beta,n.iter,checkpoint,mod.name.2,thin)}
\end{alltt}
\end{kframe}
\end{knitrout}


\subsection{Summarize results using partial data}

\begin{knitrout}
\definecolor{shadecolor}{rgb}{0.969, 0.969, 0.969}\color{fgcolor}\begin{kframe}
\begin{alltt}
\hlcom{##}
\hlcom{## Load output and calculate results}
\hlcom{##}

\hlstd{w1}\hlkwb{=}\hlkwd{new.env}\hlstd{()}
\hlkwd{load}\hlstd{(}\hlkwd{paste}\hlstd{(}\hlstr{"~/MCMCOutputModifiedDataChain1.RData"}\hlstd{,}
           \hlkwc{sep}\hlstd{=}\hlstr{""}\hlstd{),} \hlkwc{envir}\hlstd{=w1)}

\hlstd{w2}\hlkwb{=}\hlkwd{new.env}\hlstd{()}
\hlkwd{load}\hlstd{(}\hlkwd{paste}\hlstd{(}\hlstr{"~/MCMCOutputModifiedDataChain2.RData"}\hlstd{,} \hlkwc{sep}\hlstd{=}\hlstr{""}\hlstd{),}
     \hlkwc{envir}\hlstd{=w2)}
\hlstd{(status}\hlkwb{=}\hlkwd{sum}\hlstd{(}\hlopt{!}\hlkwd{is.na}\hlstd{(w2}\hlopt{$}\hlstd{out[[}\hlnum{1}\hlstd{]][,}\hlnum{1}\hlstd{])))}
\end{alltt}
\begin{verbatim}
## [1] 10000000
\end{verbatim}
\begin{alltt}
\hlstd{thin}\hlkwb{=}\hlnum{1000} \hlcom{# large thinning value required due to autocorrelation}
\hlstd{burn}\hlkwb{=}\hlnum{100000}
\hlstd{ind}\hlkwb{=}\hlkwd{seq}\hlstd{(burn}\hlopt{+}\hlnum{1}\hlstd{,status,thin)}
\hlkwd{length}\hlstd{(ind)}
\end{alltt}
\begin{verbatim}
## [1] 9900
\end{verbatim}
\begin{alltt}
\hlstd{N.tot1}\hlkwb{=}\hlstd{w1}\hlopt{$}\hlstd{out[[}\hlnum{4}\hlstd{]][ind]}
\hlstd{N.tot2}\hlkwb{=}\hlstd{w2}\hlopt{$}\hlstd{out[[}\hlnum{4}\hlstd{]][ind]}
\hlstd{Mean.N}\hlkwb{=}\hlkwd{mean}\hlstd{(N.tot1)}
\hlstd{LB}\hlkwb{=}\hlkwd{quantile}\hlstd{(N.tot1,}\hlnum{0.025}\hlstd{)}
\hlstd{UB}\hlkwb{=}\hlkwd{quantile}\hlstd{(N.tot1,}\hlnum{0.975}\hlstd{)}

\hlcom{##}
\hlcom{## Gelman Rubin Diagnostic}
\hlcom{##}

\hlstd{mcmc1.tmp}\hlkwb{=}\hlkwd{cbind}\hlstd{(w1}\hlopt{$}\hlstd{out[[}\hlnum{1}\hlstd{]][}\hlnum{1}\hlopt{:}\hlstd{status,],w1}\hlopt{$}\hlstd{out[[}\hlnum{3}\hlstd{]][}\hlnum{1}\hlopt{:}\hlstd{status,])}
\hlstd{mcmc2.tmp}\hlkwb{=}\hlkwd{cbind}\hlstd{(w2}\hlopt{$}\hlstd{out[[}\hlnum{1}\hlstd{]][}\hlnum{1}\hlopt{:}\hlstd{status,],w2}\hlopt{$}\hlstd{out[[}\hlnum{3}\hlstd{]][}\hlnum{1}\hlopt{:}\hlstd{status,])}
\hlstd{chain1}\hlkwb{=}\hlkwd{mcmc}\hlstd{(mcmc1.tmp)}
\hlstd{chain2}\hlkwb{=}\hlkwd{mcmc}\hlstd{(mcmc2.tmp)}
\hlstd{out.list}\hlkwb{=}\hlkwd{mcmc.list}\hlstd{(chain1,chain2)}
\hlstd{(GR.Diag}\hlkwb{=}\hlkwd{gelman.diag}\hlstd{(out.list,}\hlkwc{confidence} \hlstd{=} \hlnum{0.95}\hlstd{,}
                     \hlkwc{transform}\hlstd{=}\hlnum{FALSE}\hlstd{,}\hlkwc{autoburnin}\hlstd{=}\hlnum{TRUE}\hlstd{)[}\hlnum{2}\hlstd{])}
\end{alltt}
\begin{verbatim}
## $mpsrf
## [1] 1.003851
\end{verbatim}
\begin{alltt}
\hlcom{##}
\hlcom{## Effective sample size}
\hlcom{##}

\hlstd{(ESS}\hlkwb{=}\hlkwd{min}\hlstd{(}\hlkwd{ess}\hlstd{(chain1)))}
\end{alltt}
\begin{verbatim}
## [1] 3465.526
\end{verbatim}
\begin{alltt}
\hlcom{##}
\hlcom{## Bayesian p-value}
\hlcom{##}

\hlstd{T.mcmc.chi2}\hlkwb{=}\hlstd{w2}\hlopt{$}\hlstd{out[[}\hlnum{5}\hlstd{]][ind]}
\hlstd{T.data.chi2}\hlkwb{=}\hlstd{w2}\hlopt{$}\hlstd{out[[}\hlnum{6}\hlstd{]][ind]}
\hlstd{(Bayes.p}\hlkwb{=}\hlkwd{sum}\hlstd{(T.mcmc.chi2}\hlopt{>=}\hlstd{T.data.chi2,}
             \hlkwc{na.rm}\hlstd{=}\hlnum{TRUE}\hlstd{)}\hlopt{/}\hlkwd{length}\hlstd{(ind))}
\end{alltt}
\begin{verbatim}
## [1] 0.5633333
\end{verbatim}
\begin{alltt}
\hlcom{##}
\hlcom{## Sampled posterior predictive value}
\hlcom{##}

\hlkwd{set.seed}\hlstd{(}\hlnum{2017}\hlstd{)}
\hlstd{n}\hlkwb{=}\hlkwd{dim}\hlstd{(Y)[}\hlnum{1}\hlstd{]}
\hlstd{J}\hlkwb{=}\hlkwd{dim}\hlstd{(Y)[}\hlnum{2}\hlstd{]}
\hlstd{param.vec.id}\hlkwb{=}\hlkwd{sample}\hlstd{(ind,}\hlnum{1}\hlstd{)}
\hlstd{p.sppv}\hlkwb{=}\hlstd{w1}\hlopt{$}\hlstd{out[[}\hlnum{1}\hlstd{]][param.vec.id]}
\hlstd{lambda.sppv}\hlkwb{=}\hlstd{w1}\hlopt{$}\hlstd{out[[}\hlnum{3}\hlstd{]][param.vec.id,]}
\hlstd{N.sppv}\hlkwb{=}\hlstd{w1}\hlopt{$}\hlstd{out[[}\hlnum{2}\hlstd{]][param.vec.id,]}
\hlstd{Expected.Y}\hlkwb{=}\hlkwd{matrix}\hlstd{(N.sppv}\hlopt{*}\hlstd{p.sppv,}\hlkwd{dim}\hlstd{(Y)[}\hlnum{1}\hlstd{],}\hlkwd{dim}\hlstd{(Y)[}\hlnum{2}\hlstd{])}
\hlstd{reps}\hlkwb{=}\hlnum{100000}
\hlstd{T.mcmc.sppv.chi2}\hlkwb{=}\hlkwd{numeric}\hlstd{(reps)}
\hlstd{T.data.sppv.chi2}\hlkwb{=}\hlkwd{numeric}\hlstd{(reps)}
\hlkwa{for}\hlstd{(k} \hlkwa{in} \hlnum{1}\hlopt{:}\hlstd{reps)\{}
    \hlstd{y.sppv}\hlkwb{=}\hlkwd{matrix}\hlstd{(}\hlkwd{rbinom}\hlstd{(n}\hlopt{*}\hlstd{J,N.sppv,p.sppv),n,J)}
    \hlstd{y.sppv[}\hlkwd{is.na}\hlstd{(Y)]}\hlkwb{=}\hlnum{NA}
    \hlcom{## T.mcmc.sppv.chi2[k]=sum(apply(y.sppv,1,var,na.rm=TRUE))}
    \hlcom{## T.data.sppv.chi2[k]=sum(apply(Y,1,var,na.rm=TRUE))}
    \hlstd{T.mcmc.sppv.chi2[k]}\hlkwb{=}\hlkwd{sum}\hlstd{(((y.sppv}\hlopt{-}\hlstd{Expected.Y)}\hlopt{^}\hlnum{2}\hlstd{)}\hlopt{/}\hlstd{Expected.Y,}\hlkwc{na.rm}\hlstd{=}\hlnum{TRUE}\hlstd{)}
    \hlstd{T.data.sppv.chi2[k]}\hlkwb{=}\hlkwd{sum}\hlstd{(((Y}\hlopt{-}\hlstd{Expected.Y)}\hlopt{^}\hlnum{2}\hlstd{)}\hlopt{/}\hlstd{Expected.Y,}\hlkwc{na.rm}\hlstd{=}\hlnum{TRUE}\hlstd{)}
\hlstd{\}}
\hlstd{(sppv}\hlkwb{=}\hlkwd{sum}\hlstd{(T.mcmc.sppv.chi2}\hlopt{>=}\hlstd{T.data.sppv.chi2)}\hlopt{/}\hlstd{reps)}
\end{alltt}
\begin{verbatim}
## [1] 0.82266
\end{verbatim}
\end{kframe}
\end{knitrout}


\subsection{Plot MCMC output}



\begin{center}
  \begin{figure}[H]
    \includegraphics[width=.9\linewidth,keepaspectratio]{foo.pdf}
    \caption[Convergence and posterior distributions -- modified
    data]{Convergence diagnostics and posterior distributions of model
    parameters using modified sea otter data.}
  \end{figure}
\end{center}



\begin{center}
  \begin{figure}[H]
    \includegraphics[width=.9\linewidth,keepaspectratio]{expectedobserved2.pdf}
    \caption[Observed vs. expected -- modified data]{Observed
      vs. expected counts of sea otters at 20 sites in Glacier Bay,
      AK, based on model fit after removing one site that violated the
      closure assumption. Black line: posterior distribution of
      expected counts. Red lines: sum of the observed counts for two
      observation periods (i.e., sum of column 1 and column 2 of \texttt{SeaOtterData[-19,]}).}
  \end{figure}
\end{center}



\end{document}

